%
% Copyright 2014, NICTA
%
% This software may be distributed and modified according to the terms of
% the BSD 2-Clause license. Note that NO WARRANTY is provided.
% See "LICENSE_BSD2.txt" for details.
%
% @TAG(NICTA_BSD)
%

This chapter will provide the interfaces which make up the \refOS protocol. These interfaces will
set the language with which components in the system will then interact. An interface will also aim
to provide the abstraction to manage conceptual objects, such as a process or a file or a device.

Implementations may choose to extend these interfaces to provide any extra system functionality, as
well as attach additional interfaces.

% ----------------------------------------------------------------------
\section{Objects}

Servers implement interfaces, which provide management (i.e. creation,
destruction, monitoring, sharing, manipulation) object(s). For example, an
audio device may implement the dataspace interface for UNIX \texttt{/dev/audio},
which gives the abstraction to manage "audio" objects.

In our sample implementation, the access to an object (i.e. the permission to
invoke the methods which manage the object) is implemented using a badged
endpoint capability. The badge number is used to uniquely identify the instance
of the object. Since most expected implementations of the \refOS protocol will
use a similar method, the term \texttt{cap} (capability) may be used
interchangably with \texttt{object}. For example "window cap" means "a
capability which represents the access to a window object."

Note that in implementations, objects may be merged in some cases. For example, the process
server's \obj{liveness}, \obj{anon} or \obj{process} object caps may be merged into one for
simplification.

\begin{description}

	\item[\obj{process}] is an object, most likely maintained by the process
	server (i.e. \srv{procserv}), which represents a process. If a something
	has access to a process object, then it has the access to kill that
	process, call methods on behalf of that process...etc.
      
	\item[\obj{liveness}] is an object representing the identity of a process.
	If something has access to a liveness object of a process, then it can be
	notified of the process's death can ask for an ID to uniquely identify the
	process, but cannot kill the process or pretend to be it.
       
	\item[\obj{anon}] is an object representing the "address" of a server, used
	to establish a session connection to the server.
    
	\item[\obj{session}] is an object representing an active connection session
	to a server.  If something has access to a session object, then it can
	invoke methods on the server on behalf of the session client.
    
	\item[\obj{dataspace}] is an object that represents a dataspace. The
	dataspace itself may represent anything that may be modeled as a series of
	bytes including devices, RAM or files.  If something has access to a
	dataspace object, then it may read from / write to / execute it (depending
	on the dataspace permissions) by mapping a memory window to the dataspace,
	or close / delete the dataspace.
      
	\item[\obj{memwindow}] is an object representing an address space range
	(i.e. a memory window) segment in a process's virtual memory. If something has
	access to a memory window object, then it may map it to a dataspace, map a
	frame into it, and so forth.
    
\end{description}

% ----------------------------------------------------------------------
\section{Methods}

% ------------------
\subsection{Process Server Interface}

The process server interface aims to provide the abstraction for managing everything to do with a
process and its threads. This includes management of the process's virtual memory, memory window
creation / deletion, process creation / deletion, death notification of a client process, thread
management, and so on. Note that in implementations, the \cp{procserv}{session} may be the same
capability as \cp{procserv}{process} (in which case the process server would be connectionless).
It may also be shared with \cp{procserv}{anon} for simplification.

\begin{description}
	\item \pro{procserv}{session}{watch\_client(\cp{procserv}{liveness}, death\_notify\_ep)} 
	   {(Errorcode, death\_id, principle\_id)}   
    Authenticate a new client of a server against the \srv{procserv} and register for death
    notification.
    \begin{description}
        \item [procserv\_liveness\_C] The new client's liveness cap, which they have given to the
                server through session connection.
        \item [death\_notify\_ep] The async endpoint through which death notification will happen.
        \item [death\_id] The unique clientID that server will receive on death notification.
        \item [principle\_id] Optional, may be used for permission checking.
    \end{description}

    \item \pro{procserv}{session}{unwatch\_client(\cp{procserv}{liveness})}{(Errorcode)}
    Stop watching a client and remove death notifications about it.

	\item \pro{procserv}{session}{create\_mem\_window(base\_vaddr, size, permissions, flags)}
	   {(ErrorCode, \cp{procserv}{window})}
	Create a new memory window segment for the calling client.
	Note that clients may only create memory windows for their own address space, and alignment
	restrictions may apply here due to implementation and/or hardware restrictions. In the \refOS
    client environment, a valid memory window segment must be covering any virtual address ranges
    before any mapping can be done (including dataspace \& device frame mappings).
    \begin{description}
        \item [base\_vaddr] The window base address in the calling client's VSpace.
        \item [size] The size of the mem window.
        \item [permissions] Optional, read / write permission flags.
        \item [flags] Optional, extra flags bitmask (eg. cached / uncached window).
    \end{description}

    \item \pro{procserv}{session}{resize\_memory\_window(\cp{procserv}{window}, size)}
        {ErrorCode}
    Resize a memory window segment. Optional luxy feature, may be useful for implementing
    dynamic heap on clients.
    \begin{description}
        \item [size] The new size of the mem window to resize to.
    \end{description}

    \item \pro{procserv}{session}{delete\_memory\_window(\cp{procserv}{window})}
        {ErrorCode}
    Delete a memory window segment.

	\item \pro{procserv}{session}{register\_as\_pager(\cp{procserv}{window}, fault\_notify\_ep)} 
	   {(ErrorCode, window\_id)} 
    Register to receive faults for the presented window. The returned \ty{window\_id} is an integer
    used during notification in order for the pager to be able to identify which window faulted.
    \ty{window\_id} must be unique for each pager, although may also be implemented to be unique
    across the entire system.
    \begin{description}
        \item [fault\_notify\_ep] The async endpoint which the fault notifications will
                be sent through.
        \item [window\_id] The uniqie ID of the window, which will be used to identify which window
                faulted. The server most likely will have to book-keep this ID to handle faults
                correctly.
    \end{description}

    \item \pro{procserv}{session}{unregister\_as\_pager(\cp{procserv}{window})}{(ErrorCode)} 
    Unregister to stop being the pager for a client process's memory window.

	\item \pro{procserv}{session}{window\_map(\cp{procserv}{window}, window\_offset, src\_addr)}
	   {ErrorCode}
	This syscall is most commonly used in response to a prior fault notification from the
    process server. Maps the frame at the given VSpace into the client's faulted window, and
    then resolves the fault and resumes execution of the faulting client. Also may be used to
    eagerly map frames into clients before they VMfault there.
    \begin{description}
        \item [window\_offset] The offset into the window to map the frame into.
        \item [src\_addr] The address of the source frame in the calling client process's own
                VSpace; this address should contain a valid frame, and page-alignment restrictions
                may apply for this parameter.
    \end{description}
	
	\item \pro{procserv}{session}{new\_proc(name, params, block, priority)}{(ErrorCode, status)}
	Starts a new process, blocking or non-blocking.
    \begin{description}
        \item [name] The executable file name of the process to start.
        \item [params] The parameters to pass onto the new process.
        \item [block] Whether to block until the child process exits.
        \item [priority] The priority of the new child process.
        \item [status] Output, containint the exit status of the process. Only applicable if
                       blocking.
    \end{description}

    \item \pro{procserv}{session}{exit(status)}{(ErrorCode)}
    Exits and deletes the calling process.
    \begin{description}
        \item [status] The exit status of the calling client process.
    \end{description}

    \item \pro{procserv}{session}{clone(entry, stack, flags, args)}{(ErrorCode, thread\_id)}
    Starts a new thread, sharing the current client process' address space. The child
    thread will have the same priority as the parent process.
    \begin{description}
        \item [entry] The entry point vaddr of the new thread.
        \item [stack] The stack vaddr of the new thread.
        \item [flags] Any thread-related flags.
        \item [args] Thread arguments.
        \item [thread\_id] The thread ID of the cloned thread.
    \end{description}

\end{description}

% ------------------
\subsection{Server Connection Interface}

The server connection interface provides client session connection and disconnection. It may be a
good idea during implementation to extend this interface onto any other extra OS functioanlity that
is common across servers eg. debug ping, parameter buffer setup, notification buffer setup, and so
forth.

\begin{description}

    \item \pro{serv}{anon}{connect(\cp{procserv}{liveness})}{(ErrorCode, \cp{serv}{session})}
        Connect to a server and establish a session.

    \item \pro{serv}{session}{disconnect()}{(ErrorCode)}
        Disconnect from a server and delete session.

\end{description}

% ------------------
\subsection{Data Server Interface}

The data server interface provides the abstraction for management of dataspaces, including
creation, access, sharing, manipulation...etc.

\begin{description}
      
    \item \pro{dataserv}{session}{open(char *name, flags, mode, size)}
        {(ErrorCode, \cp{dataserv}{dataspace})}
    Opens a new dataspace at the dataspace server, which represents a series of bytes. Dataspace
    mapping methods such as datamap() and init\_data() directly or indirectly maps the contents of
    this dataspace into a memory window, after which the contents can be read from / written to.
    The concept of a dataspace in \refOS is similar to a file in UNIX; what a dataspace
    represents depends on the server that is implementing this interface.
    \begin{description}
        \item [name] The name of the dataspace to open.
        \item [flags] The read / write / create flags.
        \item [mode] The mode to create new file with, in the case that a new one is created.
        \item [size] The size of dataspace to open. Note that some data servers may ignore this.
    \end{description}

    \item \pro{dataserv}{session}{close(\cp{dataserv}{dataspace})}{ErrorCode}
    Close a dataspace belonging to the data server.

    \item \pro{dataserv}{session}{expand(\cp{dataserv}{dataspace}, size)}{ErrorCode}
    Resize and expand a given dataspace. Note that some dataspaces may not support this, as
    sometimes the size of a dataspaces makes no sense (eg. serial input).
    \begin{description}
        \item [size] The new size to expand dataspace to.
    \end{description}

    \item \pro{dataserv}{session}{datamap(\cp{dataserv}{dataspace}, \cp{procserv}{window}, offset)}
        {ErrorCode}
    Request that the data server back the specified window with the specified dataspace. Offset is the
    offset into the dataspace to be mapped to the start of the window. Note: the dataspace has to be
    provided by the session used to request the backing of the window.
    \begin{description}
        \item [procserv\_window\_C] Cap to the memory window to map the dataspace contents into.
        \item [offset] The offset in bytes from the beginning of the dataspace.
    \end{description}

    \item \pro{dataserv}{session}{dataunmap(\cp{procserv}{window})}
        {ErrorCode}
    Unmap the contents of the data from the given memory window.
    \begin{description}
        \item [procserv\_window\_C] Capability to window to unmap dataspace from.
    \end{description}

    \item \pro{dataserv}{session}{init\_data(\cp{dest}{dataspace}, \cp{dataserv}{dataspace},
        offset)}{ErrorCode}
    Initialises another dataspace by the contents of a source dataspace. The source dataspace is
    where the content is, and must originate from the invoked dataserver. Whether the destination
    dataspace and the source dataspace coming from the same dataserver is supported is up to the
    dataserver implementation; the dataserver documentation should be referred to. One example use
    case for this is a memory manager implementing the dataspace for RAM having a block of ram
    initialised by an external data source such as a file from a file server.
    \begin{description}
        \item [dest\_dataspace\_C] Another dataspace to be initialised with content from the 
                                 source dataspace.
        \item [dataserv\_dataspace\_C] The dataserver's own dataspace, where the content is.
        \item [offset] The content offset into the source dataspace.
    \end{description}

    \item \pro{dataserv}{session}{have\_data(\cp{dataserv}{dataspace}, fault\_ep)}
        {ErrorCode, data\_id}
    The local dataspace server calls another remote dataspace server to have the remote dataspace
    initalised by the contents of the local dataspace. The local dataspace must book-keep the given
    data\_id. The remote dataspace then will notify through the given endpoint asking for content
    initialisation, the dataID being given in the notification.
    \begin{description}
        \item [fault\_ep] The async endpoint to ask for content initialisation with.
        \item [data\_id] The remote endpoint's unique data ID number, used for notification.
    \end{description}

    \item \pro{dataserv}{session}{unhave\_data(\cp{dataserv}{dataspace})}{ErrorCode}
    Tell dataserver to stop providing content initialise data for its dataspace.

    \item \pro{dataserv}{session}{provide\_data(\cp{dataserv}{dataspace}, offset, content\_size)}
        {ErrorCode}  
    Gives the content from the local dataserver to the remote dataserver, in response to the
    remote server's earlier notification asking for content. The content is assumed to be in
    the set up parameter buffer. This call implicitly requires a parameter buffer to be
    set up, how this is done is up to the implementation.
    Note that even though the notification from the dataserver asking for content uses dataID to
    identify which dataspace, the reply, for security reasons, gives the actual dataspace cap.
    The dataID may be used securely iff the dataserver implementation supports per-client dataID
    checking, and a version of this method with dataID instead of a cap may be added.
    \begin{description}
        \item [offset] The offset into the remote dataspace to provide content for.
        \item [content\_size] The size of the content.
    \end{description}

    \item \pro{dataserv}{process}{datashare(\cp{dataserv}{dataspace})}{ErrorCode}
    Share a dataspace of a dataserver with another process. The exact implementation of this will be
    context-based. For example, a server sharing a parameter buffer may implement this function as
    \texttt{share\_param\_buffer(dataspace\_cap)}, implicitly stating the context, and implementing
    multiple share calls for more contexts. Although this method of stating the sharing context is
    strongly encouraged, the exact method of passing context is left up to the implementation. \\
    Note: There are a few assumptions made here.
    \begin{itemize}
    \item The dataspace is backed by somebody the receiving process trusts
        (eg. the process server).
    \item The dataspace is not revocable so the receiving process does
        not need to protect itself from an untrusted fault handler on that memory.
    \end{itemize}

    \item \pro{dataserv}{event}{pagefault\_notify(window\_id, offset, op)}{ErrorCode} \\
    Send a notification event to a data server indicating which window the fault occurred in,
    the offset within that window, and operation attempted (either Read or Write).

    \item \pro{dataserv}{event}{initdata\_notify(data\_id, offset)}{ErrorCode} \\
    Send a notification event to a data server indicating which dataspace needs its initial data.

    \item \pro{dataserv}{event}{death\_notify(death\_id)}{ErrorCode}
    Send a notification event to a data server indicating a client of
    theirs has died and the resources associated with them can be cleaned up.
  
\end{description}

% ------------------
\subsection{Name Server Interface}

The name server interface provides a common interface for servers to implement. We have chose a
simple hierachical distributed naming scheme here, where each server is responsible for a particular
prefix under another name server. This allows for simple distributed naming while allowing prefix
caching. A root name server must be defined from which naming starts; in our sample implementation
the process server acts as the root name server.

\begin{description}

    \item \pro{nameserv}{session}{register(name, \cp{dataserv}{anon})} {ErrorCode} \\
    Servers need to create an endpoint and call this on the root name server, in order for clients
    to be able to find this server and to connect to it. The anon cap will be given to clients
    looking for this server, and the client will make its connection call through the anon cap
    to establish a session.
    Re-registering will replace the current server anon cap.
    \begin{description}
        \item [name] The name to register under.
        \item [dataserv\_anon\_C] The anonymous endpoint to register with.
    \end{description}

    \item \pro{nameserv}{session}{unregister(name)} {ErrorCode} \\
    Unregister as a server under given name, so clients are no longer able to find us
    under that name. Existing anon caps will be invalidated.
    \begin{description}
        \item [name] The name to unregister for.
    \end{description}

    \item\pro{nameserv}{session}{resolve\_segment(path)}
        {(ErrorCode, \cp{dataserv}{anon}, resolved\_bytes)}
    Gives back an anon cap if the server is found. For details regarding the anon cap, refer to
    the documentation under register(). This function resolves a part of the name, returned
    as offset into the string that has been resolved. The rest of the path string may be
    resolved by another dataspace, and so forth, until we get to the end containing the endpoint
    to the server that actually contains the file. This allows for a simple hierachical
    namespace with distributed naming.
    \begin{description}
        \item [path] The path to resolve.
        \item [resolved\_bytes] The number of bytes resolved from start of path string.
    \end{description}

\end{description}

% ----------------------------------------------------------------------
\section{Server Components}

RefOS embraces the component-based multi-server OS design. The two main components in our design are
the process server and the file server. In our sample implementation we included an additoinal OS
server which is a dataserver responsible for basic device related tasks.

% ------------------
\subsection{Process Server}

The process server is the most trusted component in the system. It runs as the initial kernel thread
and does not depend on any other component (this avoids deadlock). The process server implementation
is single-threaded. The process server also implements the dataspace interface for anonymous memory,
and acts as the memory manager.

In our implementation, the process server implements the following interfaces:
\begin{itemize}
    \item Process server interface. (naming, mem. windows, processes..etc)
    \item Dataspace server interface. (For anonymous memory)
    \item Name server interface (In our implementation, acts as the root name server)
\end{itemize}

% ------------------
\subsection{File Server}

The file server is more trusted than clients, but less trusted than the process server (avoids
deadlock). In our current implementation the file server does not use a disk driver, and the actual
file contents are compiled into the file server executable itself using a pre-compiled archive.
The file server acts as the main data server in our system.

In our implementation, the file server implements the following interfaces.
\begin{itemize}
    \item Dataspace server interface. (for stored file data)
    \item Server connection interface. (for clients to connect to it)
\end{itemize}

% ------------------
\subsection{Console Server}

The Console server provides serial / EGA input \& output functionality, exposed through
the dataspace interface. It also provides terminal emulation for EGA screen output.

In our implementation, the Console server implements the following interfaces.

\begin{itemize}
    \item Dataspace server interface. (for serial input / output, and EGA screen devices)
    \item Server connection interface. (for clients to connect to it)
\end{itemize}

% ------------------
\subsection{Timer Server}

The Timer server provides timer gettime \& sleep functionality, exposed through
the dataspace interface.

In our implementation, the Timer server implements the following interfaces.

\begin{itemize}
    \item Dataspace server interface. (for timer devices)
    \item Server connection interface. (for clients to connect to it)
\end{itemize}
